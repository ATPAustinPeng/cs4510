\documentclass[11pt,a4paper]{article}

% ----- preamble.tex -----
% % ----- preamble.tex -----
% % ----- preamble.tex -----
% \input{preamble.tex} should be included in all note files before '\begin{document}'
% \lecture{<lectureNumber>: <lectureTopic>}{<mm/dd/yyyy>}{<lecturerName>}{<authorName>}

\hbadness=10000
\vbadness=10000

% page size settings
\setlength{\textheight}{8.5in}
\setlength{\textwidth}{6.0in}
\setlength{\headheight}{0in}
\addtolength{\topmargin}{-.5in}
\addtolength{\oddsidemargin}{-.5in}

% paragraph spacing/indenting settings
\setlength\parindent{0pt}
\setlength\parskip{2.5pt}

% imported packages
\usepackage{amsmath,amssymb,amsthm}
\usepackage{mdframed} % for boxing
\usepackage{graphicx} % for inserting images
\graphicspath{ {./imgs/} }

\usepackage{diagbox} % backslash for tabular/tables
\usepackage{booktabs} % enhances quality of table presentation
\usepackage{tikz}
\usetikzlibrary{arrows,automata,trees,calc}

% ----- \tableofcontents PAGE LINKING (without blue highlight) -----
\usepackage{hyperref}
\hypersetup{
    colorlinks,
    citecolor=black,
    filecolor=black,
    linkcolor=black,
    urlcolor=black
}

% ----- VARIABLE RE-DEFINITIONS -----
\def\epsilon{\varepsilon}
\def\phi{\varphi}


% ----- TODO: MAKE EVERY NEW SECTION START ON NEW PAGE EXCLUDING TOC -----
% \let\oldsection\section
% \renewcommand{\section}{\newpage\oldsection}


% ----- CUSTOM COMMANDS -----
\newcommand{\handout}[5]{
    % \renewcommand{\thepage}{#1-\arabic{page}}
    \noindent
    \begin{center}
        \framebox{
            \vbox{
                \hbox to 5.78in {{\bf CS 4510: Automata and Complexity}\hfill #2}
                \vspace{4mm}
                \hbox to 5.78in {{\Large \hfill #5  \hfill}}
                \vspace{2mm}
                \hbox to 5.78in {{\it #3 \hfill #4}}
            }
        }
   \end{center}
   \vspace*{4mm}
}

\newcommand{\lecture}[4]{\handout{#1}{#2}{Lecturer: #3}{Author: #4}{Lecture #1}}

% keeps contents of the same theorem on the same page
\newcommand{\blocktheorem}[1]{
    \csletcs{old#1}{#1}
    \csletcs{endold#1}{end#1}
    \RenewDocumentEnvironment{#1}{o}{
        % \par\addvspace{cm}
        \noindent\begin{minipage}{\textwidth}
        \IfNoValueTF{##1}{\csuse{old#1}}{\csuse{old#1}[##1]}}{\csuse{endold#1}
        \end{minipage}
        \par\addvspace{0.75cm}
    }
}
\raggedbottom

\theoremstyle{definition}
\newtheorem{example}{Example} % use for example problems
\newmdtheoremenv{definition}{Definition} % use for definitions
\newtheorem{theorem}{Theorem} % use for theorems
\newtheorem{corollary}{Corollary} % use for corollary
\newtheorem{lemma}{Lemma} % use for corollary
\newtheorem{claim}{Claim}

\blocktheorem{example}
\blocktheorem{definition}
\blocktheorem{theorem}
\blocktheorem{corollary}
\blocktheorem{lemma}
\blocktheorem{claim} should be included in all note files before '\begin{document}'
% \lecture{<lectureNumber>: <lectureTopic>}{<mm/dd/yyyy>}{<lecturerName>}{<authorName>}

\hbadness=10000
\vbadness=10000

% page size settings
\setlength{\textheight}{8.5in}
\setlength{\textwidth}{6.0in}
\setlength{\headheight}{0in}
\addtolength{\topmargin}{-.5in}
\addtolength{\oddsidemargin}{-.5in}

% paragraph spacing/indenting settings
\setlength\parindent{0pt}
\setlength\parskip{2.5pt}

% imported packages
\usepackage{amsmath,amssymb,amsthm}
\usepackage{mdframed} % for boxing
\usepackage{graphicx} % for inserting images
\graphicspath{ {./imgs/} }

\usepackage{diagbox} % backslash for tabular/tables
\usepackage{booktabs} % enhances quality of table presentation
\usepackage{tikz}
\usetikzlibrary{arrows,automata,trees,calc}

% ----- \tableofcontents PAGE LINKING (without blue highlight) -----
\usepackage{hyperref}
\hypersetup{
    colorlinks,
    citecolor=black,
    filecolor=black,
    linkcolor=black,
    urlcolor=black
}

% ----- VARIABLE RE-DEFINITIONS -----
\def\epsilon{\varepsilon}
\def\phi{\varphi}


% ----- TODO: MAKE EVERY NEW SECTION START ON NEW PAGE EXCLUDING TOC -----
% \let\oldsection\section
% \renewcommand{\section}{\newpage\oldsection}


% ----- CUSTOM COMMANDS -----
\newcommand{\handout}[5]{
    % \renewcommand{\thepage}{#1-\arabic{page}}
    \noindent
    \begin{center}
        \framebox{
            \vbox{
                \hbox to 5.78in {{\bf CS 4510: Automata and Complexity}\hfill #2}
                \vspace{4mm}
                \hbox to 5.78in {{\Large \hfill #5  \hfill}}
                \vspace{2mm}
                \hbox to 5.78in {{\it #3 \hfill #4}}
            }
        }
   \end{center}
   \vspace*{4mm}
}

\newcommand{\lecture}[4]{\handout{#1}{#2}{Lecturer: #3}{Author: #4}{Lecture #1}}

% keeps contents of the same theorem on the same page
\newcommand{\blocktheorem}[1]{
    \csletcs{old#1}{#1}
    \csletcs{endold#1}{end#1}
    \RenewDocumentEnvironment{#1}{o}{
        % \par\addvspace{cm}
        \noindent\begin{minipage}{\textwidth}
        \IfNoValueTF{##1}{\csuse{old#1}}{\csuse{old#1}[##1]}}{\csuse{endold#1}
        \end{minipage}
        \par\addvspace{0.75cm}
    }
}
\raggedbottom

\theoremstyle{definition}
\newtheorem{example}{Example} % use for example problems
\newmdtheoremenv{definition}{Definition} % use for definitions
\newtheorem{theorem}{Theorem} % use for theorems
\newtheorem{corollary}{Corollary} % use for corollary
\newtheorem{lemma}{Lemma} % use for corollary
\newtheorem{claim}{Claim}

\blocktheorem{example}
\blocktheorem{definition}
\blocktheorem{theorem}
\blocktheorem{corollary}
\blocktheorem{lemma}
\blocktheorem{claim} should be included in all note files before '\begin{document}'
% \lecture{<lectureNumber>: <lectureTopic>}{<mm/dd/yyyy>}{<lecturerName>}{<authorName>}

\hbadness=10000
\vbadness=10000

% page size settings
\setlength{\textheight}{8.5in}
\setlength{\textwidth}{6.0in}
\setlength{\headheight}{0in}
\addtolength{\topmargin}{-.5in}
\addtolength{\oddsidemargin}{-.5in}

% paragraph spacing/indenting settings
\setlength\parindent{0pt}
\setlength\parskip{2.5pt}

% imported packages
\usepackage{amsmath,amssymb,amsthm}
\usepackage{mdframed} % for boxing
\usepackage{graphicx} % for inserting images
\graphicspath{ {./imgs/} }

\usepackage{diagbox} % backslash for tabular/tables
\usepackage{booktabs} % enhances quality of table presentation
\usepackage{tikz}
\usetikzlibrary{arrows,automata,trees,calc}

% ----- \tableofcontents PAGE LINKING (without blue highlight) -----
\usepackage{hyperref}
\hypersetup{
    colorlinks,
    citecolor=black,
    filecolor=black,
    linkcolor=black,
    urlcolor=black
}

% ----- VARIABLE RE-DEFINITIONS -----
\def\epsilon{\varepsilon}
\def\phi{\varphi}


% ----- TODO: MAKE EVERY NEW SECTION START ON NEW PAGE EXCLUDING TOC -----
% \let\oldsection\section
% \renewcommand{\section}{\newpage\oldsection}


% ----- CUSTOM COMMANDS -----
\newcommand{\handout}[5]{
    % \renewcommand{\thepage}{#1-\arabic{page}}
    \noindent
    \begin{center}
        \framebox{
            \vbox{
                \hbox to 5.78in {{\bf CS 4510: Automata and Complexity}\hfill #2}
                \vspace{4mm}
                \hbox to 5.78in {{\Large \hfill #5  \hfill}}
                \vspace{2mm}
                \hbox to 5.78in {{\it #3 \hfill #4}}
            }
        }
   \end{center}
   \vspace*{4mm}
}

\newcommand{\lecture}[4]{\handout{#1}{#2}{Lecturer: #3}{Author: #4}{Lecture #1}}

% keeps contents of the same theorem on the same page
\newcommand{\blocktheorem}[1]{
    \csletcs{old#1}{#1}
    \csletcs{endold#1}{end#1}
    \RenewDocumentEnvironment{#1}{o}{
        % \par\addvspace{cm}
        \noindent\begin{minipage}{\textwidth}
        \IfNoValueTF{##1}{\csuse{old#1}}{\csuse{old#1}[##1]}}{\csuse{endold#1}
        \end{minipage}
        \par\addvspace{0.75cm}
    }
}
\raggedbottom

\theoremstyle{definition}
\newtheorem{example}{Example} % use for example problems
\newmdtheoremenv{definition}{Definition} % use for definitions
\newtheorem{theorem}{Theorem} % use for theorems
\newtheorem{corollary}{Corollary} % use for corollary
\newtheorem{lemma}{Lemma} % use for corollary
\newtheorem{claim}{Claim}

\blocktheorem{example}
\blocktheorem{definition}
\blocktheorem{theorem}
\blocktheorem{corollary}
\blocktheorem{lemma}
\blocktheorem{claim}

\begin{document}
\lecture{3: Operations On Languages}{08/30/2022}{Zvi Galil}{Austin Peng}
\tableofcontents

% TODO: how do i fix this to have new page before every section, but not before the table of contents
\AddToHook{cmd/section/before}{\newpage}

\section{Operations On Languages}
\subsection{The Regular Operations: Union, Concatenation, Kleene Star}
\begin{definition}
    Let $A$ and $B$ be languages. We define the regular operations union, concatenation, and Kleene star as follows:

    \begin{itemize}
        \item \textbf{union}: $A\cup B = \{x \mid x\in A\text{ or }x\in B\}$
        \item \textbf{concatenation}: $A\circ B = \{xy\mid x\in A\text{ and }y\in B\}$
        \item \textbf{Kleene star}: $A^* = \{x_1x_2 ... x_n\mid k\geq 0 \text{ and each } x_i\in A\}$
        \begin{itemize}
            \item $A^* = \bigcup\limits_{k\geq 0}A^k$
        \end{itemize}
    \end{itemize}

    Note: $A^+=A^*-\{\epsilon\}$.
\end{definition}

\begin{example}
    $$A=\{\text{good},\text{bad}\},B=\{\text{boy},\text{girl}\}$$
    $$A\cup B=\{\text{good},\text{bad},\text{boy},\text{girl}\}$$
    $$A\circ B=\{\text{goodboy},\text{goodgirl},\text{badboy},\text{badgirl}\}$$
    $$A^*=\{\epsilon,\text{good},\text{bad},\text{goodgood},\text{goodbad},\text{badgood},\text{badbad},...\}$$
\end{example}

\begin{theorem}
    The class of regular languages is closed under the union operation. That is, if $A$ and $B$ are regular languages, so is $A\cup B$.

    \begin{proof}
        Let $M_1$ recognize $A_1$, where $M_1=(Q_1,\Sigma,\delta_1,q_1,F_1)$ and $M_2$ recognize $A_2$, where $M_2=(Q_2,\Sigma,\delta_2,q_2,F_2)$. \\

        Construct $M$ to recognize $A_1\cup A_2$, where $M=(Q,\Sigma,\delta,q,F)$.
        \begin{enumerate}
            \item $Q=\{(r_1,r_2)\mid r_1\in Q_1\text{ and }r_2\in Q_2\}$
            \begin{itemize}
                \item this set $Q$ is essentially $Q_1\times Q_2$, the Cartesian product of sets $Q_1$ and $Q_2$
                \item it is the set of all pairs of states: the first from $Q_1$ and the second from $Q_2$
            \end{itemize}
            \item $\Sigma=\Sigma$, the same as in $M_1$ and $M_2$
            \begin{itemize}
                \item for simplicity, assume $M_1$ and $M_2$ have the same alphabet $\Sigma$
                \item the theorem remains true if they have different alphabets $\Sigma_1$ and $\Sigma_2$
                \item then modify the proof to let $\Sigma=\Sigma_1\cup\Sigma_2$
            \end{itemize}
            \item $\delta$ (transition function) is defined as follows:
                \begin{itemize}
                    \item for each $(r_1,r_2)\in Q$ and each $a\in\Sigma$, let $\delta((r_1,r_2),a)=(\delta_1(r_1,a),\delta_2(r_2,a))$
                \end{itemize}
            \item $q_0$ is the pair $(q_1,q_2)$
            \item $F$ is the set of pairs in which either member is an accepting state of $M_1$ or $M_2$ as follows:
            \begin{itemize}
                \item $F=\{(r_1,r_2)\mid r_1\in F_1\text{ or }r_2\in F_2\}$
                \item the above expression is the same as $F=(F_1\times Q_2)\cup(Q_1\times F_2)$
                \item Note: $F=(F_1\times F_2)$ IS NOT CORRECT! (takes intersection instead of union)
            \end{itemize}
        \end{enumerate}
    \end{proof}
\end{theorem}


\section{Other Regular Operations}
\subsection{Complement}
\begin{theorem}
    The class of regular languages is closed under the complement operation.

    \begin{proof}
        Let $L$ be a regular language, then some finite automaton $M$ recognizes $L$. \\

        Let $\overline{M}$ be the same as $M$, but with the accepting and non-accepting states interchanged.
        Then $\overline{M}$ accepts a string $x$ if and only if $M$ does not accept $x$. So, $L(\overline{M})=\overline{L}$. \\
    \end{proof}
\end{theorem}

\subsection{Intersection}
\begin{theorem}
    If $A_1$ and $A_2$ are regular languages, then so is $A_1\cap A_2$.

    \begin{proof}
        Let $M_1=(Q^1,\Sigma,\delta^1,q_0^1,F^1)$ decide $A_1$ and $M_2=(Q^2,\Sigma,\delta^2,q_0^2,F^2)$ decide $A_2$. \\

        We construct the automaton $M=(Q,\Sigma,\delta,q_0,F)$ as follows:
        \begin{itemize}
            \item $Q=Q^1\times Q^2$ (each state in $M$ is a pair of states in $M_1$ and $M_2$)
            \item $\Sigma$ is the same shared alphabet as $M_1$ and $M_2$
            \item $\delta((r_1,r_2),x)=(\delta^1(r_1,x),\delta^2(r_2,x))$
            \item $q_0=(q_0^1,q_0^2)$
            \item $F=F^1\times F^2$ (both $M_1$ and $M_2$ must be in an accepting state for $M$ to accept)
        \end{itemize}
    \end{proof}
\end{theorem}

\subsection{Set Difference}
\begin{theorem}
    If $A$ and $B$ are regular languages, then so is $A_1 \backslash A_2=\{x \mid x\in A\text{ and } x\notin B\}$.

    \begin{proof}
        Note: $A\backslash B=A\cap\overline{B}$ \\

        Since regular languages are closed under intersection and complement, regular languages are closed under subtraction. \\
    \end{proof}
\end{theorem}

\subsection{Symmetric Difference}
\begin{theorem}
    If $A$ and $B$ are regular langauges, then so is $A\oplus B$.

    \begin{proof}
        Note: $A\oplus B=(A\cup B)\backslash(A\cap B)$ \\

        Since regular languages are closed under union, intersection, and subtraction, regular languages are closed under symmetric difference. \\
    \end{proof}
\end{theorem}

\section{Closed Operations}
A set $S$ is closed under an operation $\cdot$ if for every $a,b\in S,a\cdot b\in S$.
That is, if we apply the operation to any two element in the set, we another element in the same set.

\begin{itemize}
    \item $\mathbb{N}$ is closed under addition
    \item $\mathbb{N}$ is not closed under subtraction (ex. $3-5=-2\notin\mathbb{N})$
    \item $\mathbb{Z}$ is closed under addition, subtraction, multiplication, but not division.
    \item $\mathbb{Q}$ is closed under addition, subtraction, multiplication, and $\mathbb{Q} \backslash \{0\}$ is closed under division.
    \item $\mathbb{R}$ is not closed under square root, but $\mathbb{R^+}$ is closed under square root.
    \begin{itemize}
        \item $\mathbb{R}$ has negative numbers, while $\mathbb{R^+}$ does not.
    \end{itemize}
    \item $\mathbb{C}$ is closed under square root (because it includes imaginary numbers).
\end{itemize}


\end{document}