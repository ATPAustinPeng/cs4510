\documentclass[11pt,a4paper]{article}

% ----- preamble.tex -----
% % ----- preamble.tex -----
% % ----- preamble.tex -----
% \input{preamble.tex} should be included in all note files before '\begin{document}'
% \lecture{<lectureNumber>: <lectureTopic>}{<mm/dd/yyyy>}{<lecturerName>}{<authorName>}

\hbadness=10000
\vbadness=10000

% page size settings
\setlength{\textheight}{8.5in}
\setlength{\textwidth}{6.0in}
\setlength{\headheight}{0in}
\addtolength{\topmargin}{-.5in}
\addtolength{\oddsidemargin}{-.5in}

% paragraph spacing/indenting settings
\setlength\parindent{0pt}
\setlength\parskip{2.5pt}

% imported packages
\usepackage{amsmath,amssymb,amsthm}
\usepackage{mdframed} % for boxing
\usepackage{graphicx} % for inserting images
\graphicspath{ {./imgs/} }

\usepackage{diagbox} % backslash for tabular/tables
\usepackage{booktabs} % enhances quality of table presentation
\usepackage{tikz}
\usetikzlibrary{arrows,automata,trees,calc}

% ----- \tableofcontents PAGE LINKING (without blue highlight) -----
\usepackage{hyperref}
\hypersetup{
    colorlinks,
    citecolor=black,
    filecolor=black,
    linkcolor=black,
    urlcolor=black
}

% ----- VARIABLE RE-DEFINITIONS -----
\def\epsilon{\varepsilon}
\def\phi{\varphi}


% ----- TODO: MAKE EVERY NEW SECTION START ON NEW PAGE EXCLUDING TOC -----
% \let\oldsection\section
% \renewcommand{\section}{\newpage\oldsection}


% ----- CUSTOM COMMANDS -----
\newcommand{\handout}[5]{
    % \renewcommand{\thepage}{#1-\arabic{page}}
    \noindent
    \begin{center}
        \framebox{
            \vbox{
                \hbox to 5.78in {{\bf CS 4510: Automata and Complexity}\hfill #2}
                \vspace{4mm}
                \hbox to 5.78in {{\Large \hfill #5  \hfill}}
                \vspace{2mm}
                \hbox to 5.78in {{\it #3 \hfill #4}}
            }
        }
   \end{center}
   \vspace*{4mm}
}

\newcommand{\lecture}[4]{\handout{#1}{#2}{Lecturer: #3}{Author: #4}{Lecture #1}}

% keeps contents of the same theorem on the same page
\newcommand{\blocktheorem}[1]{
    \csletcs{old#1}{#1}
    \csletcs{endold#1}{end#1}
    \RenewDocumentEnvironment{#1}{o}{
        % \par\addvspace{cm}
        \noindent\begin{minipage}{\textwidth}
        \IfNoValueTF{##1}{\csuse{old#1}}{\csuse{old#1}[##1]}}{\csuse{endold#1}
        \end{minipage}
        \par\addvspace{0.75cm}
    }
}
\raggedbottom

\theoremstyle{definition}
\newtheorem{example}{Example} % use for example problems
\newmdtheoremenv{definition}{Definition} % use for definitions
\newtheorem{theorem}{Theorem} % use for theorems
\newtheorem{corollary}{Corollary} % use for corollary
\newtheorem{lemma}{Lemma} % use for corollary
\newtheorem{claim}{Claim}

\blocktheorem{example}
\blocktheorem{definition}
\blocktheorem{theorem}
\blocktheorem{corollary}
\blocktheorem{lemma}
\blocktheorem{claim} should be included in all note files before '\begin{document}'
% \lecture{<lectureNumber>: <lectureTopic>}{<mm/dd/yyyy>}{<lecturerName>}{<authorName>}

\hbadness=10000
\vbadness=10000

% page size settings
\setlength{\textheight}{8.5in}
\setlength{\textwidth}{6.0in}
\setlength{\headheight}{0in}
\addtolength{\topmargin}{-.5in}
\addtolength{\oddsidemargin}{-.5in}

% paragraph spacing/indenting settings
\setlength\parindent{0pt}
\setlength\parskip{2.5pt}

% imported packages
\usepackage{amsmath,amssymb,amsthm}
\usepackage{mdframed} % for boxing
\usepackage{graphicx} % for inserting images
\graphicspath{ {./imgs/} }

\usepackage{diagbox} % backslash for tabular/tables
\usepackage{booktabs} % enhances quality of table presentation
\usepackage{tikz}
\usetikzlibrary{arrows,automata,trees,calc}

% ----- \tableofcontents PAGE LINKING (without blue highlight) -----
\usepackage{hyperref}
\hypersetup{
    colorlinks,
    citecolor=black,
    filecolor=black,
    linkcolor=black,
    urlcolor=black
}

% ----- VARIABLE RE-DEFINITIONS -----
\def\epsilon{\varepsilon}
\def\phi{\varphi}


% ----- TODO: MAKE EVERY NEW SECTION START ON NEW PAGE EXCLUDING TOC -----
% \let\oldsection\section
% \renewcommand{\section}{\newpage\oldsection}


% ----- CUSTOM COMMANDS -----
\newcommand{\handout}[5]{
    % \renewcommand{\thepage}{#1-\arabic{page}}
    \noindent
    \begin{center}
        \framebox{
            \vbox{
                \hbox to 5.78in {{\bf CS 4510: Automata and Complexity}\hfill #2}
                \vspace{4mm}
                \hbox to 5.78in {{\Large \hfill #5  \hfill}}
                \vspace{2mm}
                \hbox to 5.78in {{\it #3 \hfill #4}}
            }
        }
   \end{center}
   \vspace*{4mm}
}

\newcommand{\lecture}[4]{\handout{#1}{#2}{Lecturer: #3}{Author: #4}{Lecture #1}}

% keeps contents of the same theorem on the same page
\newcommand{\blocktheorem}[1]{
    \csletcs{old#1}{#1}
    \csletcs{endold#1}{end#1}
    \RenewDocumentEnvironment{#1}{o}{
        % \par\addvspace{cm}
        \noindent\begin{minipage}{\textwidth}
        \IfNoValueTF{##1}{\csuse{old#1}}{\csuse{old#1}[##1]}}{\csuse{endold#1}
        \end{minipage}
        \par\addvspace{0.75cm}
    }
}
\raggedbottom

\theoremstyle{definition}
\newtheorem{example}{Example} % use for example problems
\newmdtheoremenv{definition}{Definition} % use for definitions
\newtheorem{theorem}{Theorem} % use for theorems
\newtheorem{corollary}{Corollary} % use for corollary
\newtheorem{lemma}{Lemma} % use for corollary
\newtheorem{claim}{Claim}

\blocktheorem{example}
\blocktheorem{definition}
\blocktheorem{theorem}
\blocktheorem{corollary}
\blocktheorem{lemma}
\blocktheorem{claim} should be included in all note files before '\begin{document}'
% \lecture{<lectureNumber>: <lectureTopic>}{<mm/dd/yyyy>}{<lecturerName>}{<authorName>}

\hbadness=10000
\vbadness=10000

% page size settings
\setlength{\textheight}{8.5in}
\setlength{\textwidth}{6.0in}
\setlength{\headheight}{0in}
\addtolength{\topmargin}{-.5in}
\addtolength{\oddsidemargin}{-.5in}

% paragraph spacing/indenting settings
\setlength\parindent{0pt}
\setlength\parskip{2.5pt}

% imported packages
\usepackage{amsmath,amssymb,amsthm}
\usepackage{mdframed} % for boxing
\usepackage{graphicx} % for inserting images
\graphicspath{ {./imgs/} }

\usepackage{diagbox} % backslash for tabular/tables
\usepackage{booktabs} % enhances quality of table presentation
\usepackage{tikz}
\usetikzlibrary{arrows,automata,trees,calc}

% ----- \tableofcontents PAGE LINKING (without blue highlight) -----
\usepackage{hyperref}
\hypersetup{
    colorlinks,
    citecolor=black,
    filecolor=black,
    linkcolor=black,
    urlcolor=black
}

% ----- VARIABLE RE-DEFINITIONS -----
\def\epsilon{\varepsilon}
\def\phi{\varphi}


% ----- TODO: MAKE EVERY NEW SECTION START ON NEW PAGE EXCLUDING TOC -----
% \let\oldsection\section
% \renewcommand{\section}{\newpage\oldsection}


% ----- CUSTOM COMMANDS -----
\newcommand{\handout}[5]{
    % \renewcommand{\thepage}{#1-\arabic{page}}
    \noindent
    \begin{center}
        \framebox{
            \vbox{
                \hbox to 5.78in {{\bf CS 4510: Automata and Complexity}\hfill #2}
                \vspace{4mm}
                \hbox to 5.78in {{\Large \hfill #5  \hfill}}
                \vspace{2mm}
                \hbox to 5.78in {{\it #3 \hfill #4}}
            }
        }
   \end{center}
   \vspace*{4mm}
}

\newcommand{\lecture}[4]{\handout{#1}{#2}{Lecturer: #3}{Author: #4}{Lecture #1}}

% keeps contents of the same theorem on the same page
\newcommand{\blocktheorem}[1]{
    \csletcs{old#1}{#1}
    \csletcs{endold#1}{end#1}
    \RenewDocumentEnvironment{#1}{o}{
        % \par\addvspace{cm}
        \noindent\begin{minipage}{\textwidth}
        \IfNoValueTF{##1}{\csuse{old#1}}{\csuse{old#1}[##1]}}{\csuse{endold#1}
        \end{minipage}
        \par\addvspace{0.75cm}
    }
}
\raggedbottom

\theoremstyle{definition}
\newtheorem{example}{Example} % use for example problems
\newmdtheoremenv{definition}{Definition} % use for definitions
\newtheorem{theorem}{Theorem} % use for theorems
\newtheorem{corollary}{Corollary} % use for corollary
\newtheorem{lemma}{Lemma} % use for corollary
\newtheorem{claim}{Claim}

\blocktheorem{example}
\blocktheorem{definition}
\blocktheorem{theorem}
\blocktheorem{corollary}
\blocktheorem{lemma}
\blocktheorem{claim}

\begin{document}
\lecture{1: Introduction}{08/23/2022}{Zvi Galil}{Austin Peng}
\tableofcontents

% TODO: how do i fix this to have new page before every section, but not before the table of contents
\AddToHook{cmd/section/before}{\newpage}

\section{Topics To Be Familiar With:}
\subsection{Sets}
\begin{itemize}
    \item sets are elements drawn from some universe:
    \begin{itemize}
        \item $\mathbb{Z}$ (integers)
        \item $\mathbb{N}$ (natural numbers)
        \item $\mathbb{R}$ (real numbers)
        \item $\mathbb{C}$ (complex numbers)
    \end{itemize}
    \item set operations:
    \begin{itemize}
        \item union: $A \cup B$
        \item intersection: $A \cap B$
        \item complement: $\overline{A}$
        \item Cartesian product: $A \times b$ (pairs)
        \item power: $A^k$ (k-tuples)
    \end{itemize}
\end{itemize}

\subsection{Functions}
\begin{itemize}
    \item functions are mappings from one set to another:
    \begin{itemize}
        \item $f: D \rightarrow R$
        \subitem $f(x) = x^2$
        \item $f: Z \times Z \rightarrow Z $
        \subitem $f(x,y) = x + y$
    \end{itemize}
\end{itemize}

\subsection{Relations}
\begin{itemize}
    \item relation $R$ can be written as $xRy$
    \begin{itemize}
        \item \textbf{reflexive}: $xRx$
        \item \textbf{symmetric}: $xRy \implies yRx$
        \item \textbf{transitive}: $xRy \text{ and } yRz \implies xRz$
        \item \textbf{equivalence}: if relation R is reflexive, symmetric, and transitive
    \end{itemize}
\end{itemize}

\subsection{Graphs}
\begin{itemize}
    \item a graph $G=(V,E)$ has vertices $V$ and edges $E$
    \item a graph may be directed or undirected
\end{itemize}

\subsection{Strings}
\begin{itemize}
    \item "abc" is a string
    \item a string is a sequence of symbols from an alphabet $\Sigma$
    \item $\Sigma^*$ is the set of all strings over $\Sigma$
    \item $\Sigma^*$ contains the empty string $\epsilon$, all strings of length 1, all strings of length 2, etc.
\end{itemize}

\subsection{Boolean Logic}
\begin{itemize}
    \item 3 main operations: "and" ($\land$), "or" ($\lor$), "not" ($\lnot$)
    \item these operations are performed on variables and constants (true and false)
\end{itemize}

\subsection{Proofs}
\begin{itemize}
    \item mathematics consists of definitions, statements/lemmas/theorems, and proofs
    \item 3 common types of proofs:
    \begin{itemize}
        \item proof by construction
        \item proof by contradiction
        \item proof by induction
    \end{itemize}
    \item these operations are performed on variables and constants (true and false)
\end{itemize}

\section{Finite Automata and (Regular Languages)}
Consider an automatic door with the inputs $\{\text{F(ront)}, \text{N(either)}, \text{B(oth)}, \text{R(ear)}\}$.
The door has 2 states: open and closed. The door will only open if there is someone in front and not in the rear (door swings inward).
If the door is open, but nobody is there, the door will close.

\begin{tikzpicture}[node distance={3cm}, semithick, main/.style = {draw, circle}]
    \node[state] (0) {closed};
    \node[state] (1) [right of=0] {open};

    % closed
    \path (0) edge [loop left] node {N,B,R} (0);
    \draw[->,out=15,in=165] (0) to node[above] {F} (1);

    % open
    \path (1) edge [loop right] node {F,B,R} (1);
    \draw[->,out=195,in=345] (1) to node[below] {N} (0);
\end{tikzpicture}

\begin{example} $M_1$

    \begin{tikzpicture}[node distance={3cm}, semithick, main/.style = {draw, circle}]
        \node[name=input] {};
        \node[state] (0) [right of=input]{$q_0$};
        \node[state,accepting] (1) [right of=0] {$q_1$};
        \node[state] (2) [right of=1] {$q_2$};

        % q_0
        \draw[->] (input) -- (0);
        \path (0) edge [loop above] node {0} (0);
        \draw[->] (0) to node[above] {1} (1);

        % q_1
        \path (1) edge [loop above] node {1} (1);
        \draw[->,out=15,in=165] (1) to node[above] {1} (2);
        
        % q_2
        \draw[->,out=195,in=345] (2) to node[below] {0, 1} (1);
    \end{tikzpicture}
\end{example}

\begin{itemize}
    \item $q_0$ is the starting state
    \item $q_1$ is the accepting statements
    \item all transitions ($\delta=0,1$) are defined for all states
    \item let x be a binary string $1101$, $M_1$ will accept $x$ because it ends at $q_1$
    \item $L(M_1)=$ all $x$ such s.t $M_1$ accepts $x$ if $x$ has a 1 AND either ends with a 1 or with an even number of 0s
\end{itemize}

\subsection{Formal Definition Of A Deterministic Finite Automaton}
\begin{definition}
    A deterministic finite automaton is a 5-tuple $M=(Q, \Sigma, \delta, q_0, F)$ where:
    \begin{itemize}
        \item $Q$ is a finite set called the states
        \item $\Sigma$ is a finite set called the alphabet
        \item $\delta: Q\times\Sigma\rightarrow Q$ is the transition function
        \item $q_0$ is the initial state
        \item $F\subseteq Q$ is the accepting states
    \end{itemize}
    Note: $\delta(q,a)=p$ means that if DFA is in state $q$ and sees $a$, it goes to state $p$
\end{definition}

\newpage
\subsection{Formal Definition Of Computation}
\begin{definition}
    A \textbf{computation} of a deterministic finite automaton (DFA) $M$ on an input string $x=x_1,...,x_n\in\Sigma^*$ is a sequence of states $q_0,...,q_n\in Q$ s.t. for $i=0,...,n-1$ $\delta(q_i,x_{i+1})=q_{i+1}$.
    We say that $M$ accepts $x$ if $q_n\in F$.
    If $q_n\notin F$, then we say $M$ rejects $x$.
\end{definition}

\subsection{Formal Definition Of A Language}
\begin{definition}
    The \textbf{language} of $M$ is defined as the set $\{x\in\Sigma^*: M\text{ accepts }x\}$.
    A language is a set of strings.
\end{definition}

\begin{example}
    $\emptyset,\{\epsilon\}$ are both langauges.
\end{example}

\begin{definition}
    A language $L$ is called \textbf{regular} if there is a DFA accepting $L$.
\end{definition}

\begin{example}
    $L_w=\{w\}$ is regular. $L'_w=\{\text{all strings that end in }w\}$ is also regular.
\end{example}

\end{document}