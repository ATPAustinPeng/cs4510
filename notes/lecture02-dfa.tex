\documentclass[11pt,a4paper]{article}

% ----- preamble.tex -----
% % ----- preamble.tex -----
% % ----- preamble.tex -----
% \input{preamble.tex} should be included in all note files before '\begin{document}'
% \lecture{<lectureNumber>: <lectureTopic>}{<mm/dd/yyyy>}{<lecturerName>}{<authorName>}

\hbadness=10000
\vbadness=10000

% page size settings
\setlength{\textheight}{8.5in}
\setlength{\textwidth}{6.0in}
\setlength{\headheight}{0in}
\addtolength{\topmargin}{-.5in}
\addtolength{\oddsidemargin}{-.5in}

% paragraph spacing/indenting settings
\setlength\parindent{0pt}
\setlength\parskip{2.5pt}

% imported packages
\usepackage{amsmath,amssymb,amsthm}
\usepackage{mdframed} % for boxing
\usepackage{graphicx} % for inserting images
\graphicspath{ {./imgs/} }

\usepackage{diagbox} % backslash for tabular/tables
\usepackage{booktabs} % enhances quality of table presentation
\usepackage{tikz}
\usetikzlibrary{arrows,automata,trees,calc}

% ----- \tableofcontents PAGE LINKING (without blue highlight) -----
\usepackage{hyperref}
\hypersetup{
    colorlinks,
    citecolor=black,
    filecolor=black,
    linkcolor=black,
    urlcolor=black
}

% ----- VARIABLE RE-DEFINITIONS -----
\def\epsilon{\varepsilon}
\def\phi{\varphi}


% ----- TODO: MAKE EVERY NEW SECTION START ON NEW PAGE EXCLUDING TOC -----
% \let\oldsection\section
% \renewcommand{\section}{\newpage\oldsection}


% ----- CUSTOM COMMANDS -----
\newcommand{\handout}[5]{
    % \renewcommand{\thepage}{#1-\arabic{page}}
    \noindent
    \begin{center}
        \framebox{
            \vbox{
                \hbox to 5.78in {{\bf CS 4510: Automata and Complexity}\hfill #2}
                \vspace{4mm}
                \hbox to 5.78in {{\Large \hfill #5  \hfill}}
                \vspace{2mm}
                \hbox to 5.78in {{\it #3 \hfill #4}}
            }
        }
   \end{center}
   \vspace*{4mm}
}

\newcommand{\lecture}[4]{\handout{#1}{#2}{Lecturer: #3}{Author: #4}{Lecture #1}}

% keeps contents of the same theorem on the same page
\newcommand{\blocktheorem}[1]{
    \csletcs{old#1}{#1}
    \csletcs{endold#1}{end#1}
    \RenewDocumentEnvironment{#1}{o}{
        % \par\addvspace{cm}
        \noindent\begin{minipage}{\textwidth}
        \IfNoValueTF{##1}{\csuse{old#1}}{\csuse{old#1}[##1]}}{\csuse{endold#1}
        \end{minipage}
        \par\addvspace{0.75cm}
    }
}
\raggedbottom

\theoremstyle{definition}
\newtheorem{example}{Example} % use for example problems
\newmdtheoremenv{definition}{Definition} % use for definitions
\newtheorem{theorem}{Theorem} % use for theorems
\newtheorem{corollary}{Corollary} % use for corollary
\newtheorem{lemma}{Lemma} % use for corollary
\newtheorem{claim}{Claim}

\blocktheorem{example}
\blocktheorem{definition}
\blocktheorem{theorem}
\blocktheorem{corollary}
\blocktheorem{lemma}
\blocktheorem{claim} should be included in all note files before '\begin{document}'
% \lecture{<lectureNumber>: <lectureTopic>}{<mm/dd/yyyy>}{<lecturerName>}{<authorName>}

\hbadness=10000
\vbadness=10000

% page size settings
\setlength{\textheight}{8.5in}
\setlength{\textwidth}{6.0in}
\setlength{\headheight}{0in}
\addtolength{\topmargin}{-.5in}
\addtolength{\oddsidemargin}{-.5in}

% paragraph spacing/indenting settings
\setlength\parindent{0pt}
\setlength\parskip{2.5pt}

% imported packages
\usepackage{amsmath,amssymb,amsthm}
\usepackage{mdframed} % for boxing
\usepackage{graphicx} % for inserting images
\graphicspath{ {./imgs/} }

\usepackage{diagbox} % backslash for tabular/tables
\usepackage{booktabs} % enhances quality of table presentation
\usepackage{tikz}
\usetikzlibrary{arrows,automata,trees,calc}

% ----- \tableofcontents PAGE LINKING (without blue highlight) -----
\usepackage{hyperref}
\hypersetup{
    colorlinks,
    citecolor=black,
    filecolor=black,
    linkcolor=black,
    urlcolor=black
}

% ----- VARIABLE RE-DEFINITIONS -----
\def\epsilon{\varepsilon}
\def\phi{\varphi}


% ----- TODO: MAKE EVERY NEW SECTION START ON NEW PAGE EXCLUDING TOC -----
% \let\oldsection\section
% \renewcommand{\section}{\newpage\oldsection}


% ----- CUSTOM COMMANDS -----
\newcommand{\handout}[5]{
    % \renewcommand{\thepage}{#1-\arabic{page}}
    \noindent
    \begin{center}
        \framebox{
            \vbox{
                \hbox to 5.78in {{\bf CS 4510: Automata and Complexity}\hfill #2}
                \vspace{4mm}
                \hbox to 5.78in {{\Large \hfill #5  \hfill}}
                \vspace{2mm}
                \hbox to 5.78in {{\it #3 \hfill #4}}
            }
        }
   \end{center}
   \vspace*{4mm}
}

\newcommand{\lecture}[4]{\handout{#1}{#2}{Lecturer: #3}{Author: #4}{Lecture #1}}

% keeps contents of the same theorem on the same page
\newcommand{\blocktheorem}[1]{
    \csletcs{old#1}{#1}
    \csletcs{endold#1}{end#1}
    \RenewDocumentEnvironment{#1}{o}{
        % \par\addvspace{cm}
        \noindent\begin{minipage}{\textwidth}
        \IfNoValueTF{##1}{\csuse{old#1}}{\csuse{old#1}[##1]}}{\csuse{endold#1}
        \end{minipage}
        \par\addvspace{0.75cm}
    }
}
\raggedbottom

\theoremstyle{definition}
\newtheorem{example}{Example} % use for example problems
\newmdtheoremenv{definition}{Definition} % use for definitions
\newtheorem{theorem}{Theorem} % use for theorems
\newtheorem{corollary}{Corollary} % use for corollary
\newtheorem{lemma}{Lemma} % use for corollary
\newtheorem{claim}{Claim}

\blocktheorem{example}
\blocktheorem{definition}
\blocktheorem{theorem}
\blocktheorem{corollary}
\blocktheorem{lemma}
\blocktheorem{claim} should be included in all note files before '\begin{document}'
% \lecture{<lectureNumber>: <lectureTopic>}{<mm/dd/yyyy>}{<lecturerName>}{<authorName>}

\hbadness=10000
\vbadness=10000

% page size settings
\setlength{\textheight}{8.5in}
\setlength{\textwidth}{6.0in}
\setlength{\headheight}{0in}
\addtolength{\topmargin}{-.5in}
\addtolength{\oddsidemargin}{-.5in}

% paragraph spacing/indenting settings
\setlength\parindent{0pt}
\setlength\parskip{2.5pt}

% imported packages
\usepackage{amsmath,amssymb,amsthm}
\usepackage{mdframed} % for boxing
\usepackage{graphicx} % for inserting images
\graphicspath{ {./imgs/} }

\usepackage{diagbox} % backslash for tabular/tables
\usepackage{booktabs} % enhances quality of table presentation
\usepackage{tikz}
\usetikzlibrary{arrows,automata,trees,calc}

% ----- \tableofcontents PAGE LINKING (without blue highlight) -----
\usepackage{hyperref}
\hypersetup{
    colorlinks,
    citecolor=black,
    filecolor=black,
    linkcolor=black,
    urlcolor=black
}

% ----- VARIABLE RE-DEFINITIONS -----
\def\epsilon{\varepsilon}
\def\phi{\varphi}


% ----- TODO: MAKE EVERY NEW SECTION START ON NEW PAGE EXCLUDING TOC -----
% \let\oldsection\section
% \renewcommand{\section}{\newpage\oldsection}


% ----- CUSTOM COMMANDS -----
\newcommand{\handout}[5]{
    % \renewcommand{\thepage}{#1-\arabic{page}}
    \noindent
    \begin{center}
        \framebox{
            \vbox{
                \hbox to 5.78in {{\bf CS 4510: Automata and Complexity}\hfill #2}
                \vspace{4mm}
                \hbox to 5.78in {{\Large \hfill #5  \hfill}}
                \vspace{2mm}
                \hbox to 5.78in {{\it #3 \hfill #4}}
            }
        }
   \end{center}
   \vspace*{4mm}
}

\newcommand{\lecture}[4]{\handout{#1}{#2}{Lecturer: #3}{Author: #4}{Lecture #1}}

% keeps contents of the same theorem on the same page
\newcommand{\blocktheorem}[1]{
    \csletcs{old#1}{#1}
    \csletcs{endold#1}{end#1}
    \RenewDocumentEnvironment{#1}{o}{
        % \par\addvspace{cm}
        \noindent\begin{minipage}{\textwidth}
        \IfNoValueTF{##1}{\csuse{old#1}}{\csuse{old#1}[##1]}}{\csuse{endold#1}
        \end{minipage}
        \par\addvspace{0.75cm}
    }
}
\raggedbottom

\theoremstyle{definition}
\newtheorem{example}{Example} % use for example problems
\newmdtheoremenv{definition}{Definition} % use for definitions
\newtheorem{theorem}{Theorem} % use for theorems
\newtheorem{corollary}{Corollary} % use for corollary
\newtheorem{lemma}{Lemma} % use for corollary
\newtheorem{claim}{Claim}

\blocktheorem{example}
\blocktheorem{definition}
\blocktheorem{theorem}
\blocktheorem{corollary}
\blocktheorem{lemma}
\blocktheorem{claim}

\begin{document}
\lecture{2: Deterministic Finite Automata}{08/25/2022}{Zvi Galil}{Austin Peng}
\tableofcontents

% TODO: how do i fix this to have new page before every section, but not before the table of contents
\AddToHook{cmd/section/before}{\newpage}

\section{DFA Examples}
\begin{example} $M_2$

    \begin{tikzpicture}[node distance={3cm}, semithick, main/.style = {draw, circle}]
        \node[name=input] {};
        \node[state] (0) [right of=input] {$q_0$};

        % q_0
        \draw[->] (input) -- (0);
        \path (0) edge [loop above] node {0,1} (0);
    \end{tikzpicture}

    $L(M_1)=\phi$. The language is empty because there are no accepting states.
\end{example}

\begin{example} $M_2$

    \begin{tikzpicture}[node distance={3cm}, semithick, main/.style = {draw, circle}]
        \node[name=input] {};
        \node[state,accepting] (0) [right of=input] {$q_0$};
        \node[state] (1) [right of=0] {$q_1$};

        % q_0
        \draw[->] (input) -- (0);
        \draw[->] (0) to node[above] {0,1} (1);

        % q_1
        \path (1) edge [loop above] node {0,1} (1);
    \end{tikzpicture}

    $L(M_2)=\{\epsilon\}$.
    
    Note the difference between $M_1$ and $M_2$. They recognize different langauges.
\end{example}

\begin{example} $M_3$

    \begin{tikzpicture}[node distance={3cm}, semithick, main/.style = {draw, circle}]
        \node[name=input] {};
        \node[state] (0) [right of=input] {$q_0$};
        \node[state,accepting] (1) [right of=0] {$q_1$};

        % q_0
        \draw[->] (input) -- (0);
        \path (0) edge [loop above] node {0} (0);
        \draw[->,out=15,in=165] (0) to node[above] {1} (1);

        % q_1
        \path (1) edge [loop above] node {1} (1);
        \draw[->,out=195,in=345] (1) to node[below] {0} (0);
    \end{tikzpicture}

    $L(M_3)=\{w|w\text{ ends in a 1}\}$
\end{example}

\begin{example} $M_4$

    \begin{tikzpicture}[node distance={3cm}, semithick, main/.style = {draw, circle}]
        \node[name=input] {};
        \node[state,accepting] (0) [right of=input] {$q_0$};
        \node[state] (1) [right of=0] {$q_1$};

        % q_0
        \draw[->] (input) -- (0);
        \path (0) edge [loop above] node {0} (0);
        \draw[->,out=15,in=165] (0) to node[above] {1} (1);

        % q_1
        \path (1) edge [loop above] node {1} (1);
        \draw[->,out=195,in=345] (1) to node[below] {0} (0);
    \end{tikzpicture}

    $L(M_4)=\{\epsilon\cup\text{ strings ending with 0}\}$. Note this is the complement of $L(M_3)$.
\end{example}

\begin{example} $M_5$

    \begin{tikzpicture}[node distance={2.5cm}, semithick, main/.style = {draw, circle}]
        \node[name=input] {};
        \node[state] (0) [right of=input] {$S$};
        \node[state,accepting] (1) [above right of=0] {$r_1$};
        \node[state] (2) [right of=1] {$r_2$};
        \node[state,accepting] (3) [below right of=0] {$q_1$};
        \node[state] (4) [right of=3] {$q_2$};

        % S
        \draw[->] (input) -- (0);
        \draw[->] (0) to node[above] {b} (1);
        \draw[->] (0) to node[above] {a} (3);

        % r_1
        \path (1) edge [loop above] node {b} (1);
        \draw[->,out=25,in=155] (1) to node[above] {a} (2);

        % r_2
        \path (2) edge [loop above] node {a} (2);
        \draw[->,out=205,in=335] (2) to node[above] {b} (1);

        % q_1
        \path (3) edge [loop below] node {a} (3);
        \draw[->,out=25,in=155] (3) to node[above] {b} (4);

        % q_2
        \path (4) edge [loop below] node {b} (4);
        \draw[->,out=205,in=335] (4) to node[above] {a} (3);
        
    \end{tikzpicture}

    $L(M_5)$ is the set of all strings that start and end with the same character.
    Note: $\Sigma=\{a,b\}$
\end{example}

\section{Applications Of DFA}
\subsection{Modular Arithmetic}

Let $w\in\{0,1\}^*$ (aka any binary string). We define $\overline{w}$ to be the value of the string as a binary number.
Then, for $w\in\{0,1\}^*$ and $a\in\{0,1\}$, we have the following properties:

\begin{itemize}
    \item $\overline{a}=a$
    \item $\overline{wa}=2\overline{w}+a$
\end{itemize}

We can use a DFA to recognize modular arithmetic.
For the following example, we will consider the following transition table of $\overline{w}\text{ mod }3$.
Note that the start state of our transition table is marked with an arrow.

\begin{table}[h]
    \centering
    \begin{tabular}{|l|*{3}{c|}}\hline
    \backslashbox{$\overline{w}$ mod 3}{input $a$}
        & $\overline{w0}\text{ mod }3$ & $\overline{w1}\text{ mod }3$ & state\\\hline
        0 (state $q_0$) & 0 & 1 & $\rightarrow q_0$ \\
        1 (state $q_1$) & 2 & 0 & $q_1$ \\
        2 (state $q_2$) & 1 & 2 & $q_2$ \\
        \hline
    \end{tabular}
\end{table}

If we set the accepting state to be $q_1$ then this DFA will accept exactly those strings which are $\equiv 1\text{ modulo }3$ (aka congruent to 1 modulo 3).

\subsection{String Matching}
\subsubsection{Recognizing A Single String}
For a string $w$, we can create a DFA for the language $L_w\{w\}$ as follows:

\begin{tikzpicture}[node distance={2.5cm}, semithick, main/.style = {draw, circle}]
    \node[name=input] {};
    \node[state] (0) [right of=input] {$q_0$};
    \node[state] (1) [right of=0] {$q_1$};
    \node[state] (2) [right of=1] {$...$};
    \node[state] (3) [right of=2] {$q_{n-1}$};
    \node[state,accepting] (4) [right of=3] {$q_n$};
    \node[state] (5) [below of=2] {bad};


    % q_0
    \draw[->] (input) -- (0);
    \draw[->] (0) to node[above] {$w_1$} (1);
    \draw[->] (0) to node[left] {$\lnot w_1$} (5);

    % q_1
    % \path (1) edge [loop above] node {b} (1);
    % \draw[->,out=25,in=155] (1) to node[above] {a} (2);
    \draw[->] (1) to node[above] {$w_2$} (2);
    \draw[->] (1) to node[left] {$\lnot w_2$} (5);

    % ...
    % \path (2) edge [loop above] node {a} (2);
    % \draw[->,out=205,in=335] (2) to node[above] {b} (1);
    \draw[->] (2) to node[above] {$...$} (3);
    \draw[->] (2) to node[right] {$...$} (5);

    % q_{n-1}
    % \path (3) edge [loop below] node {a} (3);
    % \draw[->,out=25,in=155] (3) to node[above] {b} (4);
    \draw[->] (3) to node[above] {$w_{n-1}$} (4);
    \draw[->] (3) to node[right] {$\lnot w_{n-1}$} (5);
 
\end{tikzpicture}

\newpage
\subsubsection{Recognizing A Suffix}
Let $L'_w$ be the set of strings that end in $w$.
An example string from this language is $1101001\in L'_{001}$, because it ends in $001$.
We can use the following transition table:

\begin{table}[h]
    \centering
    \begin{tabular}{|l|*{2}{c|}}\hline
    \backslashbox{$Q$}{input}
        & $0$ & $1$ \\\hline
        $\rightarrow$ bad & $q_0$ & bad \\
        $q_0$ & $q_{00}$ & bad \\
        $q_{00}$ & $q_{00}$ & $q_{001}$ \\
        $q_{001}$ & $q_0$ & bad \\
        \hline
    \end{tabular}
\end{table}

We define $q_{001}$ to be our only accepting state.

In the general case, we need to keep track of the longest suffix seen so far. We will use the states $\{\text{bad},q_0,...,q_n\}$.

The DFA will be in state $q_i$ if $w_1...w_i$ is the longest suffix of the input seen so far that is a prefix of $w$.
If we are in state $q_i$, then we have to see $n-i$ more symbols until we find the string.
The transition function is defined as follows:

\begin{itemize}
    \item $\delta(q_{i-1},w_i)=q_i$
    \item $\delta(q_{i-1},a\neq w_i)=q_j$, where $w_1w_2...w_j$ is the largest prefix of $w$ that is a suffix of the current input (including $a$).
\end{itemize}



\end{document}